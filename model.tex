
\label{sec:model}

\oak\/ is unique in offering a map interface for self-managed data. %, which it stores in internal buffers.
This affects the programming model as it allows applications to  access data in \oak's buffers directly. 
\Cref{ssec:managed-API} discusses the serialization of application objects into \emph{\oak\/ buffers}.
\Cref{ssec:API}  presents 
\oak's novel \emph{zero-copy API}, which reduces the need for serialization and deserialization (and hence copying).% of accessed keys and values. 


\begin{table*}[htb]
\setlength{\abovecaptionskip}{5pt}
\setlength{\belowcaptionskip}{5pt}
\begin{tabular}{ll} 
{\bf ZeroCopyConcurrentNavigableMap} & {\bf (Legacy) ConcurrentNavigableMap}\\ 
\hline
\multicolumn{2}{c}{\emph{Queries -- get and scans}}\\
OakRBuffer get(K) & V get(K)\\
Set$\langle$OakRBuffer$\rangle$ keySet() 			 & Set$\langle$K$\rangle$ keySet()\\
Set$\langle$OakRBuffer$\rangle$ valueSet()  			 & Set$\langle$V$\rangle$ valueSet()\\
Set$\langle$OakRBuffer, OakRBuffer$\rangle$ entrySet() & Set$\langle$K, V$\rangle$ entrySet()\\
\multicolumn{2}{c}{\emph{Updates}}\\
void put(K, V) & V put(K, V)\\
void remove(K) &  V remove(K)\\
boolean putIfAbsent(K, V) & V putIfAbsent(K, V)\\
boolean computeIfPresent(K, Function(OakWBuffer))	&  {\it non-atomic} V computeIfPresent(K, Function(K,V)) \\
boolean putIfAbsentComputeIfPresent(K, V, Function(OakWBuffer)) & {\it non-atomic} V merge(K, V, Function(K,V)) \\
\hline
\end{tabular}
\caption{\oak's zero-copy API versus the legacy ConcurrentNavigableMap API. 
Key and value types are K and V, resp. 
Get and scans return OakRBuffers instead of objects. Updates
%Put, remove, and putIfAbsent 
do not return the old value in order to avoid copying. 
%The update methods  computeIfPresent and putIfAbsentComputeIfPresent  are atomic.
}
\label{alg:api}
\end{table*}


\remove{
To convert objects to buffers and vice versa, Oak applications access \emph{Oak buffers} and provide  
use \emph{serializers}, as 
Second, concurrent access to Oak-resident values and the dynamic memory usage of such values are managed via the abstraction of \emph{handles}. Handles have pluggable concurrency control and memory management and are described 
in \Cref{ssec:handles}.
Finally, we introduce in \Cref{ssec:API} a novel  
}

\subsection{\oak\/ buffers and serialization}
\label{ssec:managed-API}

% A key consideration in the design of Oak is allowing keys and values to be kept in self-managed (off-heap) memory. 
%Thus, 

Oak keys and values are variable-sized. Keys are immutable, and values may be modified. 
In contrast to Java data structures holding Java objects, Oak stores data in internal buffers. 
To convert objects (both keys and values) to and from their \emph{serialized} forms, the user must implement 
%the \algvar{OakSerializer} interfacegiven in Algorithm~\ref{alg:serialize}. This interface consists of 
a (1) serializer, (2) deserializer, and (3) serialized size calculator. % (for variable-sized keys and values).  
To allow efficient search over buffer-resident keys, the user is further required to provide 
a {comparator}.

Oak's insertions use the size calculator to deduce the amount of space to be allocated, then allocate space for the given object, 
and finally invoke the serializer to write the object to the allocated space. 
By using the user-provided serializer, we create the binary representation of the object directly into \oak's internal memory.  


\remove{
\begin{algorithm}
\footnotesize
%  // serializes the object
%   // deserializes the given byte buffer
%   // returns num of bytes needed for serializing the object
\begin{verbatim}
public interface OakSerializer<T> {
  void serialize(T source, ByteBuffer targetBuffer);
  T deserialize(ByteBuffer byteBuffer);
  int calculateSize(T object);
}
public interface OakComparator<K> {
  int compareKeys(K key1, K key2);
  int compareSerializedKeys(ByteBuffer serializedKey1, 
                            ByteBuffer serializedKey2);
  int compareSerializedKeyAndKey(ByteBuffer serializedKey, 
                                 K key);  
}
\end{verbatim}
\caption{Interface for user-provided Oak serializer and comparator.}
\label{alg:serialize}
\end{algorithm}
}


%the \algvar{OakComparator} interface, also given in Algorithm~\ref{alg:serialize}. 
%The comparator compares two keys, each of which may be provided either as a deserialized object or as a serialized buffer. 
%It determines whether they are equal, and if not, which is bigger.

%After key-value pairs are ingested,  
{\oak\/ provides \algvar{OakRBuffer}  and \algvar{OakWBuffer} abstractions for internal readable and writable buffers. 
These types are lightweight on-heap facades to off-heap storage, which provide the application with familiar managed object semantics. 
For example, they may exist arbitrarily long and the memory behind them will not be reclaimed. Furthermore, user code can access 
them without worrying about concurrent access. Since keys are immutable, they are always accessed through \algvar{OakRBuffer}s, 
whereas values can be accessed both ways.}

\remove{
Internally kept keys and values may be accessed as  buffers of  types \algvar{OakRBuffer} and \algvar{OakWBuffer}, supporting the standard  API for read-only Java ByteBuffers and writable Java ByteBuffers, respectively.
\algvar{OakWBuffer}s are accessed by user-provided lambda functions passed to Oak's  compute methods.  
User code can access the buffers without worrying about concurrent access  or dynamic memory consumption.
Oak buffer objects are intermediate (on-heap) representations of the underlying objects, 
created on-demand by operations that need access to the key or value; they are ephemeral and cease to exist once the operation is completed.
}
 
%The buffer-based direct access to serialized keys and values reduces copying and deserialization of the underlying mappings. 
 %In addition to the standard ByteBuffer API, Oak buffers manage concurrency control and dynamic memory allocation for their users. 
%Thus, user functions can run computational steps on Oak-resident values with no concern of concurrency control or memory overflow and release.
%To this end, Oak buffers use the abstraction of handles, as described in Section~\ref{ssec:handles} below.


\subsection{Zero-copy API}
\label{ssec:API}

We introduce ZeroCopyConcurrentNavigableMap, \oak's ZC API.
\Cref{alg:api}   compares it to the essential methods of the %JDK ConcurrentNavigableMap 
legacy API using  a slightly simplified Java-like syntax, neglecting some technicalities 
(e.g., the use of Collections instead of Sets in some cases). 
%very similar to the real implementation.
To use the ZC API, an application  creates a ConcurrentNavigableMap-compliant 
\oak\/ map, and accesses 
%its  ZC sibling interface 
it through the \algvar{zc()} method, e.g., calling  \algvar{map.zc().get(key)} instead of the legacy  \algvar{map.get(key)}.
 
The API is changed only in so far as to avoid copying. The  \algvar{get()} and  scans 
(\algvar{keySet()}, \algvar{valueSet()}, and \algvar{entrySet()})  return \oak\ buffers instead of Java objects,
while continuing to offer the same functionality. In particular, scans offer the Set interface
with its  standard tools such as a stream API for mapreduce-style processing~\cite{JavaStream}.  Likewise, sub-range and reverse-order views 
are provided by familiar \algvar{subMap()} and \algvar{descendingMap()} methods on Sets. 


%The sets then support the standard \algvar{subMap} and  methods for range and descending iterators.

The update methods 
% \algvar{put()}, \algvar{putIfAbsent()} and \algvar{remove()}  
differ from their legacy counterparts   
in that they do not return the old value (in order to avoid  copying it). %,  which violates Oak's zero-copy design principle. 
The last two -- \algvar{computeIfPresent()}  and  \algvar{putIfAbsentComputeIfPresent()} --  atomically update values in place. 
Both take a user (lambda) function to apply to the \algvar{OakWBuffer}  of the value mapped to the given key.
Unlike the legacy map, 
\oak\ ensures that the lambda is executed atomically, exactly once, and extends the value's memory allocation if its code so requires.
%Note also that unlike in ConcurrentNavigableMap, \oak's lambda function does not return the computed value to avoid copying.
%Rather,  it performs the computation   in-place, so the newly computed value is  accessible in the map once it returns. 


While all operations are atomic,  \algvar{get()} returns access to the same underlying memory buffer that %\algvar{compute} 
other operations update in-place, while the granularity of Oak's concurrency control is at the level of individual method calls on that buffer
(e.g., reading a single integer from it). Therefore,  buffer access methods may encounter different values -- 
and even value deletions\footnote{\small{A  \algvar{get()} method  throws a  \algvar{ConcurrentModificationException}   in case the mapping is concurrently deleted.}} -- 
when accessing a buffer %returned from the same \algvar{get} 
multiple times. This is an inevitable consequence of avoiding copying.

